Binary translation is a common method used to speed up the execution of
software components in systems that cannot run them natively.  The process
consists of dynamically creating native machine code corresponding to the
foreign binary.  While the concept is straightforward, there is a slew of ways
the construction of the translator can go awry, as will be outlined in this
paper.  We attempt to enumerate difficulties encountered while writing such an
assembler from the 6502 to the x86-64 architecture.  Examples include
dynamically rewriting the emitted code to connect basic code blocks, properly
allocating host registers to the guest code, and mixing emulation and binary
translation.  We discuss the nature of these problems, and offer advice for
anyone attempting to use this method.


