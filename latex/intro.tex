\section{Introduction}

\subsection{Arguments for Binary Translation}

Binary Translation is the process through which a binary program meant for a
specific architecture can be run natively on another system.  Normally, binary
programs that cannot be run have to be interpreted instruction by instruction
with an emulator in order to reproduce its behavior.  However, this approach is
very slow, since the amount of native instructions executed for each foreign
one can be exceedingly high.

A way to speed up the execution of programs on arbitrary architectures is using
binary translation alongside emulation.  This technique takes advantage of the
fact that most architectures are composed of common commands, like moves and
rudimentary binary operations.  Therefore, it is possible to create native
executables that perform the same function as the original program.  The amount
of code needed run is lower, and execution is faster when compared to pure emulation.

Another use case for binary translation is software virtualization.  Some
programs need to access resources that are not accessible, because they do not
exist in the system.  Alternatively, the program does not have enough
privileges to modify the component, like in the case of userspace applications
attempting to change their page table register.

Virtual Machines (VMs) are such a class of programs.  The kernel running in the
VM is designed to have full access to the hardware, which in this case includes
sensitive processor registers and emulated devices.  A translator has the
responsibility of replacing code that attempts to do such accesses with
alternative instructions.  The program
emulates the resource in such a way that the VM can keep executing, and diverts
control back to it.  Popular emulators that employ binary translation, such as
Bochs\cite{lawton1996bochs} and QEMU\cite{bellard2005qemu}, utilize exactly this kind of scheme in order to emulate devices
like disks on demand.

\subsection{Binary Translation Internals}

Binary translation, much like all forms of compilation, hinges on the concept
of basic code blocks.  Basic blocks are sequences of machine instructions that
do not contain control flow commands like jumps and branches, and are the
smallest unit that binary translation works with.  More concisely, we can only
jump to the start of basic blocks. Programs are sectioned into multiple basic
blocks, which can be organized and viewed as a graph. Where nodes are the code
blocks themselves and directed edges represent a connection (through label or pointer)
from one code block to the other.  Nodes may have multiple edges to multiple nodes.

The concept of basic blocks is useful exactly because all programs enter the 
block from the beginning and exit from the end. That means that, as long as
the sum of its operations produce the same outputs for all possible inputs,
it is possible to transform the instructions without losing program correctness.
This is the fundamental principle behind compiler optimizations.

An emulator that employs binary translation goes through the program, creating and 
then executing a sequence of native instructions corresponding to the current block.
The key insight is that program correctness is independent of the underlying
platform, and is dependent only on creating the proper output for all inputs.
Therefore, it is possible to create a native program that does exactly that, using 
the original binary as a blueprint.




